% Project 2: I/O Elevators
%
% Assignment: http://web.engr.oregonstate.edu/cgi-bin/cgiwrap/dmcgrath/classes/17F/cs444/index.cgi?hw=2
%
% File: project2.tex
%
% Group 37: Taylor Alexander Brown & Lucien Armand Tamdja Tamno
% Oregon State University
% CS 444: Operating Systems II
% Fall 2017

\documentclass[10pt,draftclsnofoot,onecolumn,journal,compsoc]{IEEEtran}
% for IEEEtran usage, see http://www.texdoc.net/texmf-dist/doc/latex/IEEEtran/IEEEtran_HOWTO.pdf

\setlength{\parindent}{0em}
\setlength{\parskip}{1em}

\usepackage[margin=0.75in]{geometry}
\usepackage{listings}
\usepackage{}

\lstset {
  basicstyle=\small\ttfamily,
%  numbers=left,
  numberstyle=\scriptsize,
  showspaces=false,
  showstringspaces=false,
  breaklines=true
}

\title{Project 2: I/O Elevators}
\author{
  \IEEEauthorblockN{Taylor Alexander Brown and Lucien Armand Tamdja Tamno} \\
  \IEEEauthorblockA{CS 444: Operating Systems II \\ Oregon State University}
}
\date{October 30, 2017}

\IEEEtitleabstractindextext{
  \begin{abstract}
  this document describes various unto how the elevator algorithm is implmented to reduce the response time on how requests can be schedule.

  \end{abstract}
}

\begin{document}

\maketitle
\IEEEdisplaynontitleabstractindextext
\IEEEpeerreviewmaketitle
\vfill

\newpage


\end{enumerate}
\tableofcontents

\newpage

\section{Introduction}
this the descriptive to the I/O scheduler algorithm on its linux version. It's the way to see how the disk arm and head move around
 to comply with requests of inputing and outputing data on the disks.
\begin{enumerate}
 
\item {What do you think the main point of this assignment is?}
\par 
the main point of this assignment is to show how using the elevator algorithm shaped and helped to write the disk scheduling algorithm, which is
to define the disk's arm and head motion as requests to read and write data. in addition, this elevator algorithm based on doubly linked list was
the better version than the FCFS algorithm. \par

\item{ How did you personally approach the problem? Design decisions, algorithm, etc}
\par
As a members of the group 37, we appraoched the problem as followed:

	\begin{itemized}

		\item the first step was to go nearby elevator and see how it works as people use it
		\item the next step was to research and read an example of Look-and C-Look algorithm
		  and then try to compare it with how elevator works.
		And the findings made out the Look-Algorithm led us to SStf (or shortest seek time first). And that was a thread that
		led us the elevator alogrithm sample and the focus point on how the i/o disk sheduling algorithm was implemented 
		and with the disk scheduling being a substantial gained time algorithm that fits the best the  disk's arm and head,
		 for various requests read and write.
		 \item the final piece being to seek the impementation in linux version.

	\end{itemized}

	\item{How did you ensure your solution was correct? Testing details, for instance}
	\par

	\item {What did you learn?}
	The very first thing i learned was the practicality of the doubly linked list and how does this played out on the Noop scheduler algorithm. 
	And dos this improve a disk's arm and head in reading and writing requests.\par

	\item 




\section{Conclusion}


\end{document}